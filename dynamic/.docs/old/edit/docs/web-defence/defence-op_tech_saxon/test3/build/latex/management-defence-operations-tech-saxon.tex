%% Generated by Sphinx.
\def\sphinxdocclass{report}
\documentclass[letterpaper,10pt,openany,oneside,english]{sphinxmanual}
\ifdefined\pdfpxdimen
   \let\sphinxpxdimen\pdfpxdimen\else\newdimen\sphinxpxdimen
\fi \sphinxpxdimen=.75bp\relax

\PassOptionsToPackage{warn}{textcomp}
\usepackage[utf8]{inputenc}
\ifdefined\DeclareUnicodeCharacter
 \ifdefined\DeclareUnicodeCharacterAsOptional
  \DeclareUnicodeCharacter{"00A0}{\nobreakspace}
  \DeclareUnicodeCharacter{"2500}{\sphinxunichar{2500}}
  \DeclareUnicodeCharacter{"2502}{\sphinxunichar{2502}}
  \DeclareUnicodeCharacter{"2514}{\sphinxunichar{2514}}
  \DeclareUnicodeCharacter{"251C}{\sphinxunichar{251C}}
  \DeclareUnicodeCharacter{"2572}{\textbackslash}
 \else
  \DeclareUnicodeCharacter{00A0}{\nobreakspace}
  \DeclareUnicodeCharacter{2500}{\sphinxunichar{2500}}
  \DeclareUnicodeCharacter{2502}{\sphinxunichar{2502}}
  \DeclareUnicodeCharacter{2514}{\sphinxunichar{2514}}
  \DeclareUnicodeCharacter{251C}{\sphinxunichar{251C}}
  \DeclareUnicodeCharacter{2572}{\textbackslash}
 \fi
\fi
\usepackage{cmap}
\usepackage[T1]{fontenc}
\usepackage{amsmath,amssymb,amstext}
\usepackage[english]{babel}
\usepackage{times}
\usepackage[Bjarne]{fncychap}
\usepackage{sphinx}

\usepackage{geometry}

% Include hyperref last.
\usepackage{hyperref}
% Fix anchor placement for figures with captions.
\usepackage{hypcap}% it must be loaded after hyperref.
% Set up styles of URL: it should be placed after hyperref.
\urlstyle{same}

\addto\captionsenglish{\renewcommand{\figurename}{Fig.}}
\addto\captionsenglish{\renewcommand{\tablename}{Table}}
\addto\captionsenglish{\renewcommand{\literalblockname}{Listing}}

\addto\captionsenglish{\renewcommand{\literalblockcontinuedname}{continued from previous page}}
\addto\captionsenglish{\renewcommand{\literalblockcontinuesname}{continues on next page}}

\addto\extrasenglish{\def\pageautorefname{page}}

\setcounter{tocdepth}{1}



\title{Operation Tech Saxon}
\date{Dec 01, 2012}
\release{0.1.0}
\author{Author(s): Sion Buckler}
\newcommand{\sphinxlogo}{\vbox{}}
\renewcommand{\releasename}{Release}
\makeindex

\begin{document}

\maketitle
\sphinxtableofcontents
\phantomsection\label{\detokenize{index::doc}}



\chapter{Release Notes and Notices}
\label{\detokenize{releasenotes:release-notes-and-notices}}\label{\detokenize{releasenotes::doc}}
This section provides information about what is new or changed, including urgent issues and documentation updates.


\section{Version 0.1.0}
\label{\detokenize{releasenotes:version-0-1-0}}
This is the first release/ draft of this technical document.


\subsection{Older Versions}
\label{\detokenize{releasenotes:older-versions}}
The table below contains information and links to, older versions of this document.


\begin{savenotes}\sphinxattablestart
\centering
\sphinxcapstartof{table}
\sphinxcaption{Table 1.0 - Older Versions of this Document}\label{\detokenize{releasenotes:id1}}
\sphinxaftercaption
\begin{tabular}[t]{|\X{25}{100}|\X{25}{100}|\X{25}{100}|\X{25}{100}|}
\hline
\sphinxstyletheadfamily 
archive date
&\sphinxstyletheadfamily 
version
&\sphinxstyletheadfamily 
description
&\sphinxstyletheadfamily 
download link
\\
\hline
YYYY-MM-DD
&
0.x.x
&
N/A
&
no older version
\\
\hline
\end{tabular}
\par
\sphinxattableend\end{savenotes}


\subsection{Version 0.0.0}
\label{\detokenize{releasenotes:version-0-0-0}}
N/A


\section{Known and Corrected Issues}
\label{\detokenize{releasenotes:known-and-corrected-issues}}
Below is a table of pending issues which have been reported to our team.
These issues will be cleared from this list as and when they are remedied.


\begin{savenotes}\sphinxattablestart
\centering
\sphinxcapstartof{table}
\sphinxcaption{Table 1.1 - Known Issues}\label{\detokenize{releasenotes:id2}}
\sphinxaftercaption
\begin{tabular}[t]{|\X{10}{100}|\X{10}{100}|\X{20}{100}|\X{60}{100}|}
\hline
\sphinxstyletheadfamily 
date
&\sphinxstyletheadfamily 
version
&\sphinxstyletheadfamily 
subject
&\sphinxstyletheadfamily 
description
\\
\hline
YYYY-MM-DD
&
0.1.0
&
Draft
&
first draft only
\\
\hline
\end{tabular}
\par
\sphinxattableend\end{savenotes}

\sphinxstylestrong{Comments} - none


\section{Recently Updated Topics}
\label{\detokenize{releasenotes:recently-updated-topics}}
Nothing significant to report


\chapter{Summary}
\label{\detokenize{summary:summary}}\label{\detokenize{summary::doc}}
“THE SIGNIFICANT PROBLEMS WE HAVE CANNOT BE SOLVED AT THE SAME LEVEL OF THINKING WHICH CREATED THEM’, \sphinxstylestrong{ALBERT EINSTEIN}

This document highlights a lawful technology strategy aimed at interconnecting the world. The approach is to transform digital communications from an affordable privilege to our free and inherent right by making the network telecom infrastructures ad supported This solution is already being systematically outlawed due to the magnitude of social change will cause. This technology continues to be developed by an alliance of constitutionally motivated British Army and civilian technicians.


\chapter{Background}
\label{\detokenize{background:background}}\label{\detokenize{background::doc}}
Operation Tech Saxon recognises that network telecom infrastructures have become reliable and wide spread enough to depend societies future upon. Perhaps even more so than the archaic solutions we are still dependant on today e.g. institutions, the monetary system and law.

There are now clear opportunities to transpose network telecoms infrastructure from a subsidiary of these social systems and into a prosperous free resource. of which social systems can re-establish themselves upon.

For over a decade web based services have proven Metcalf’s technology theory by scaling their network exponentially while their services remain free to the users. This is due to their advertising model which gives service providers a better incentive than a user based billing model.

So why has the network telecom infrastructures not moved to an ad-supported model yet? Surely this would guarentee world interconnectivity.

It’s because the technology required to develop an advertising supported, free digital communications network did not exist. Until Now.


\chapter{Introducing Wave}
\label{\detokenize{introducing-wave:introducing-wave}}\label{\detokenize{introducing-wave::doc}}
R\&D successes have now enabled the development of the worlds first network telecom advertising platform, which is due for launch in 2013 under the brand: WAVE.

Voted for the 2012 Cisco British Innovation Award and 2013 Mobile Global Award, WAVE superimposes interactive display ads onto the users web browsing experience and replaces call progress tones with audio ads, resulting in monetised voice and data traffic.

This is achieved by way of a simple firmware upgrade on the infrastructures networking and telephone exchange hardware. WAVE also provides advertisers with a self-service campaign portal which serves as a comparable alternative to radio and online advertising.

Since funding for the network can now run in parallel to service adoption, the incentive to scale the network will change, causing it to grow exponentially. This solution breaks down service adoption barriers and guarantees digital communication for the unconnected.


\chapter{Network Thinking}
\label{\detokenize{network-thinking:network-thinking}}\label{\detokenize{network-thinking::doc}}
Institutional thinking is predominantly hierarchical, bureaucratic, closed, selective and controlling. Despite this being of asset to society to date, its proven to be a limiting factor in the 21st century.

The team behind WAVE now thinks and operates in a laissez-faire, open, random and supportive way, which is proving advantageous. Decision makers at WAVE are also dynamically appointed using a real-time, influence scoring technologies.


\chapter{Resourced Based Economy}
\label{\detokenize{resource-based-economy:resourced-based-economy}}\label{\detokenize{resource-based-economy::doc}}
The monetary system was once a great analytical instrument designed to measure and balance, contribution and consumption. Individuals access to resources are reflected with currency. But this model is susceptible to hacks. Since society associate most things to money, route monetary problems are systemic.

Social networks are reducing our degrees of separation in conjunction with this new technology, debt circle cancellation technologies are now becoming possible. which will help put liquidity back into the ecomomy. Since digital currency will run parallel to digital communications, there is now a real possibility of retiring the monetary system.

Within the first of many markets WAVE has applied a technology solution which achieves the effects of Jacque Fresco’s Resource Based Economy ideology. A workable solution by which society accepts that there are enough resources for everyone who actively contributes to society. A Non-incentivised motivation to explore, create and share is expected to return to natural people.


\chapter{Rights and Freedoms}
\label{\detokenize{rights-freedoms:rights-and-freedoms}}\label{\detokenize{rights-freedoms::doc}}
There are now over a billion regulatory and statutory laws that deem what is legal. But a lawful rebellion identified that introduced and still operates unconstitutionally and fraudulently. Using the Magna Carta Charter, we’re now movement has jurisdiction was witnessing legal and lawful opt-in, opt-out juristiction.

WAVE will cause wide spread effects and regardiess of legislation, deployment is imminent. The technology has already been legislated against as it poses threat, but to whom Society itself, or simply the archaic social systems still in operation today?

There is however, still a need for jurisdiction in this new social structure to ensure it is hot abused. WAVE grants free digital communication services to citizens subject to them accessing the network by use of their social profile identities. Here in lies all of the ingredients needed to enforce jurisdiction.

Governments can evolve into ‘Gatekeeper’ by re-introducing their retractable benefits and privilege system. But this time with the access points of digital communication itself But this time it can more accurately measure and manage an individuals contribution (to society) and consumption and behaviour.

WAVE also allows real-time democracy of the resource by giving the interactive adverts network control. User’s phone / internet access could become limited or content restricted until they contribute value to society’s common goals. This is potentially an improved jurisdiction that could be welcomed.

WAVE also encourages Dr Eric Schmidt’s future vision of dual crowdsourcing. In a fully comnected world, incidents can be reported collectively and anonymously using mobile devices. In response, jurors can pass back judgments, collectively and anonymously.


\chapter{Final Thoughts}
\label{\detokenize{final-thoughts:final-thoughts}}\label{\detokenize{final-thoughts::doc}}
The technologies developed in our lifetime outperform the tools we inherited.
It’s time they outranked them.


\chapter{\sphinxstylestrong{Document Author(s):}}
\label{\detokenize{index:document-author-s}}
\sphinxstylestrong{Sion Buckler}



\renewcommand{\indexname}{Index}
\printindex
\end{document}