%% Generated by Sphinx.
\def\sphinxdocclass{report}
\documentclass[letterpaper,10pt,openany,oneside,english]{sphinxmanual}
\ifdefined\pdfpxdimen
   \let\sphinxpxdimen\pdfpxdimen\else\newdimen\sphinxpxdimen
\fi \sphinxpxdimen=.75bp\relax

\PassOptionsToPackage{warn}{textcomp}
\usepackage[utf8]{inputenc}
\ifdefined\DeclareUnicodeCharacter
 \ifdefined\DeclareUnicodeCharacterAsOptional
  \DeclareUnicodeCharacter{"00A0}{\nobreakspace}
  \DeclareUnicodeCharacter{"2500}{\sphinxunichar{2500}}
  \DeclareUnicodeCharacter{"2502}{\sphinxunichar{2502}}
  \DeclareUnicodeCharacter{"2514}{\sphinxunichar{2514}}
  \DeclareUnicodeCharacter{"251C}{\sphinxunichar{251C}}
  \DeclareUnicodeCharacter{"2572}{\textbackslash}
 \else
  \DeclareUnicodeCharacter{00A0}{\nobreakspace}
  \DeclareUnicodeCharacter{2500}{\sphinxunichar{2500}}
  \DeclareUnicodeCharacter{2502}{\sphinxunichar{2502}}
  \DeclareUnicodeCharacter{2514}{\sphinxunichar{2514}}
  \DeclareUnicodeCharacter{251C}{\sphinxunichar{251C}}
  \DeclareUnicodeCharacter{2572}{\textbackslash}
 \fi
\fi
\usepackage{cmap}
\usepackage[T1]{fontenc}
\usepackage{amsmath,amssymb,amstext}
\usepackage[english]{babel}
\usepackage{times}
\usepackage[Bjarne]{fncychap}
\usepackage{sphinx}

\usepackage{geometry}

% Include hyperref last.
\usepackage{hyperref}
% Fix anchor placement for figures with captions.
\usepackage{hypcap}% it must be loaded after hyperref.
% Set up styles of URL: it should be placed after hyperref.
\urlstyle{same}

\addto\captionsenglish{\renewcommand{\figurename}{Fig.}}
\addto\captionsenglish{\renewcommand{\tablename}{Table}}
\addto\captionsenglish{\renewcommand{\literalblockname}{Listing}}

\addto\captionsenglish{\renewcommand{\literalblockcontinuedname}{continued from previous page}}
\addto\captionsenglish{\renewcommand{\literalblockcontinuesname}{continues on next page}}

\addto\extrasenglish{\def\pageautorefname{page}}

\setcounter{tocdepth}{1}



\title{LaS Operators Guide}
\date{Oct 15, 2019}
\release{0.1.0rc1}
\author{Author(s): Wave Telecom Limited}
\newcommand{\sphinxlogo}{\vbox{}}
\renewcommand{\releasename}{Release}
\makeindex

\begin{document}

\maketitle
\sphinxtableofcontents
\phantomsection\label{\detokenize{index::doc}}



\chapter{User Guide}
\label{\detokenize{index:user-guide}}
\noindent\sphinxincludegraphics{{hardware}.png}

Welcome to the Loyverse \& ScrCpy (LaS) User Guide.
Contained in this document is clear and helpful information to assist you understand, operate and enjoy LaS.

Loyverse and ScrCpy (LaS) is a free ‘purpose built’ software solution that permits the Loyverse EPOS (Android-only) Application to run on older computers which may not support running Android applications e.g. via a virtual environment (the official solution currently being suggested by the Loyverse support team). The current recommended workaround “solution”, for non-Android Operating Systems such as Windows and Mac, is not only painfully slow to interact with, but also tedious to configure and complex to operate - not ideal for your average fast paced, service industry businesses e.g. Bars, Restaraunts etc.

As the title suggests, LaS utalises a software called ScreenCopy (ScrCpy) in order to launch a java interface window, which is not only lightweight but extreamily fast, even on older machines. Android is also hosted on a second ‘headless’ machine, not the PC. The headless (screenless) machine is a low cost, highly popular Single Board Computer called a Raspberry Pi (RPi). This Raspberry Pi connects to the PC via Ethernet. The Android Operating System, which will run the Loyverse application, also runs on the Raspberry Pi in developer mode, which permits a remote viewing feature to be utalised called Android Debugging.

So Raspberry Pi is actually running Android (or in this case a customised version entitled LineageOS), with the sole purpose of running the Loyverse Application. Whereas the PC runs Linux (Ubuntu) for bespoke printer and touchscreen LaS scripts to run, then uses the ScrCpy application as a means of remotely viewing and interacting with the Raspberry Pi, Android OS and Loyverse Application.

The final solution is a desktop icon which upon selection, simulates the Loyverse application is running on the PC, when in fact its running on a seperate machine and is only visible on the PC thanks to the Android Debugging background service and the low-latency ScrCpy java interface.

Index:


\section{Release Notes and Notices}
\label{\detokenize{releasenotes:release-notes-and-notices}}\label{\detokenize{releasenotes::doc}}
This section provides information about what is new or changed, including urgent issues, Software \& documentation updates, maintenance and new releases.
\begin{itemize}
\item {} 
The last digit(s) of the versions ID will increase to reflect the new update e.g. \sphinxcode{\sphinxupquote{0.0.X}}

\item {} 
The 2nd but last digit of the version ID will increase to reflect upgrades e.g. \sphinxcode{\sphinxupquote{0.X.0}}

\end{itemize}


\subsection{Version 0.1.0 (Alpha)}
\label{\detokenize{releasenotes:version-0-1-0-alpha}}

\subsubsection{Older Versions}
\label{\detokenize{releasenotes:older-versions}}
Below are references to older version releases and release notes:


\begin{savenotes}\sphinxattablestart
\centering
\sphinxcapstartof{table}
\sphinxcaption{Table 1.0 - archived versions of LaS}\label{\detokenize{releasenotes:id1}}
\sphinxaftercaption
\begin{tabular}[t]{|\X{20}{100}|\X{20}{100}|\X{20}{100}|\X{20}{100}|\X{20}{100}|}
\hline
\sphinxstyletheadfamily 
date
&\sphinxstyletheadfamily 
version
&\sphinxstyletheadfamily 
size
&\sphinxstyletheadfamily 
description
&\sphinxstyletheadfamily 
download link
\\
\hline
14-10-2019
&
0.1.0
&
32Gb
&
RPi Image - SD Card
&
\sphinxurl{https://linkcomingsoon.com}
\\
\hline
14-10-2019
&
0.1.0
&
40Gb
&
PC Image - USB Dongle
&
\sphinxurl{https://linkcomingsoon.com}
\\
\hline
\end{tabular}
\par
\sphinxattableend\end{savenotes}


\subsection{Known and Corrected Issues}
\label{\detokenize{releasenotes:known-and-corrected-issues}}\begin{description}
\item[{Below is a table of pending issues which have been reported to our team.}] \leavevmode
These issues will be cleared from this list as and when they are remedied.

\end{description}


\begin{savenotes}\sphinxattablestart
\centering
\sphinxcapstartof{table}
\sphinxcaption{Table 1.1 - Known Issues}\label{\detokenize{releasenotes:id2}}
\sphinxaftercaption
\begin{tabular}[t]{|\X{10}{100}|\X{10}{100}|\X{20}{100}|\X{60}{100}|}
\hline
\sphinxstyletheadfamily 
date
&\sphinxstyletheadfamily 
subject
&\sphinxstyletheadfamily 
version
&\sphinxstyletheadfamily 
description
\\
\hline
14-10-2019
&
n/a
&
0.1.0
&
Proof of Concept - See how we go
\\
\hline
\end{tabular}
\par
\sphinxattableend\end{savenotes}


\subsection{Recently Updated Topics}
\label{\detokenize{releasenotes:recently-updated-topics}}
Nothing significant to report


\section{Terminology}
\label{\detokenize{terminology:terminology}}\label{\detokenize{terminology::doc}}
In addition to the terms defined elsewhere in this User Guide, the following terms shall have the respective meanings specified below;

\sphinxstylestrong{“Raspberry Pi”} is a low cost, credit-card sized computer that plugs into a computer monitor or TV, and uses a standard keyboard and mouse. It is a capable little device that enables people of all ages to explore computing, and to learn how to program in languages like Scratch and Python.

\sphinxstylestrong{“Screen Copy (ScrCpy)”} provides display and control of Android devices connected on USB (or over TCP/IP) graphically with a mouse and your keyboard. It does not require any root access and works on GNU/Linux, Windows and MacOS. … On some devices, you also need to enable an additional option to control it using keyboard and mouse.

\sphinxstylestrong{“Ubuntu”} (pronounced oo-BOON-too) is an open source Debian-based Linux distribution. Sponsored by Canonical Ltd., Ubuntu is considered a good distribution for beginners. The operating system was intended primarily for personal computers (PCs) but it can also be used on servers.

\sphinxstylestrong{“LineageOS”} is a free and open-source operating system for set-top boxes, smartphones and tablet computers, based on the Android mobile platform.

\sphinxstylestrong{“Loyverse”} is a free, easy-to-use POS application designed for small businesses. It allows you to process transactions, keep a record of inventory and sales, and establish a customer loyalty program, all at the beautiful price of \$0.00. The name “Loyverse” is an amalgamation of the words “Loyalty” and “Universe.”

\sphinxstylestrong{“End-User/ User”} An end user is the person that a software program or hardware device is designed for. … End users are also in a separate group from the installers or administrators of the product. To simplify, the end user is the person who uses the software or hardware after it has been fully developed, marketed, and installed.

\sphinxstylestrong{“Loyverse and ScrnCpy (LaS)”} Las is a bespoke set of customisations and scripts which run ontop of Ubuntu (on the PC) and ontop of LineageOS (on the Raspberry Pi). They help the PC interface with the Raspberry Pi. LaS permits the user can launch Android from the PC by simply double clicking the Desktop shortcut.


\section{Pre-Usage Preparation}
\label{\detokenize{preperation:pre-usage-preparation}}\label{\detokenize{preperation::doc}}
This section helps prepaire the user(s) for Loyverse \& ScrCpy (LaS), although there is very little preperation and planning required with LaS, since it is designed to function as an out-of-the-box, plug \& play solution.


\subsection{Dependancies}
\label{\detokenize{preperation:dependancies}}
\sphinxstylestrong{Power} The Raspberry Pi is very specific about its power supply being 5V and 2.5A. But in the case of LaS the PC’s USB port is sufficient enough to power the Raspberry Pi.

\sphinxstylestrong{Networking} Use Wi-Fi to connect the PC to the internet, since the Ethernet port of the machine is used to communicate with the Raspberry Pi. You can scan for wireless networks to connect to, via the desktop interface, after the system has been booted up for the first time. The Raspberry Pi does has built in Wi-Fi, which you may be inclined to enable through the Android interface. \sphinxstylestrong{BUT DO NOT ENABLE THE RASPBERRY PI WI-FI. IT WILL CONFUSE THE LaS CONFIGURATION}. The LaS solution does not require an internet connection to display the Loyverse application on the PC, although Loyverse will demand internet to initially login. The PC connects to the wireless hotspot and the internet is shared to the ethernet port, which the Raspberry Pi is connected to. This is how the Raspberry Pi and Android and the Loyverse application gets its internet. Additionally the LaS solution can double-up as a source of music for your venue, which can be stored on the machine and played offline. Although music streaming over internet is a more popular option these days. The other main value of being connected to the internet is for the ongoing updates and bugfixes.

\sphinxstylestrong{Users} Users must have a basic knowledge of operating a computer e.g. selecting desktop icons to open them or checking devices such as printers are connected and powered up before trying to use them etc - It’s worth noting that the screen may lock, in which case a password will be asked for. The default system password will be provided by your systems administrator or software distributor. Basic communications skills are important when dealing with support services.

\sphinxstylestrong{Environment} The computer hardware must be kept dry and to a reasonable humidity level. If you decide to load additional applications and programes to the Raspberry Pi’s Android operating system and/or the PC’s Ubunutu operating system, please be aware of (and vigilant against) viruses. Please also be aware that there are also a number of dependant applications running on both the PC’s and Raspberry Pi’s operating system, which may cause faults if removed e.g. Android uses a startup application to open the Loyverse application automatically upon bootup.

On the PC, the LaS solution is built on the Linux Ubuntu Operating System. For support using Ubuntu please visit \sphinxurl{https://help.ubuntu.com/stable/ubuntu-help/}
On the Raspberry Pi, The LaS Solution is built upon the LineageOS (Android for Raspberry Pi) Operating System. For support using Android please visit \sphinxurl{https://wiki.lineageos.org/}


\section{Getting Started Guide}
\label{\detokenize{gettingstarted:getting-started-guide}}\label{\detokenize{gettingstarted::doc}}
\sphinxstylestrong{Setting Up LaS:} - LaS comes in two parts. An Operating System for the Raspberry Pi (RPi) and an Operating System for the Main Computer. Copy the RPi OS to its Micro SD Card and insert it. And copy the Machine OS to a USB Dongle and boot up the machine. Next connect the RPi to the computer using an Ethernet Cable and boot it up. For troubleshooting (if you run into problems) it can also be handy to have a display and mouse/keyboard for the RPi.

It takes around a minute for the RPi and machine to boot up, although the first time can take longer. Upon successful boot you will be able to connect to the Wi-Fi from the main machine as you would any normal PC. On the desktop you will see an icon, which runs a script to establish a connection with the RPi to open up Android on the Main machine using the mightly ScrCpy. This is faster than hosting Android in a virtual machine and/or trying to setup remote access.

\sphinxstylestrong{Our Hardware Solutions …} The RPi used in LaS Version 1.0 is the Raspberry Pi 3B+, although the OS should run on any RPi. The machine we’re using in Version 1.0 is an IBM (Touchscreen) SurePoint 500 POS, which has an Intel Celeron Processor and since it has no onboard Wi-Fi (and the Ethernet is being occupied by the RPi) we’ve added a Wireless Dongle (something something 88EU Chipset).


\section{Installation Guide}
\label{\detokenize{installation:installation-guide}}\label{\detokenize{installation::doc}}
Plug in, power up.


\section{Maintenance}
\label{\detokenize{productmaint:maintenance}}\label{\detokenize{productmaint::doc}}

\subsection{Software}
\label{\detokenize{productmaint:software}}
Upgrades and updates for the base operating systems (LyneageOS e.g. Android, on the Raspberry Pi and Linux Ubuntu on the PC) are set to happen automatically after the prompt to update/ upgrade is accepted.

The LaS aspects which contains custom source code for the touchscreen and connection between the two machines, is able to be remotely maintained. The source code is hosted in a public repository (GitHub) and the two machines are configured to check for updates to this source code twice daily. In the case of full LaS upgrades, the new version of the Operating System images will need to be downloaded and then loaded onto the machines e.g. copying to the SD Card/ USB Dongle and inserted into the Raspberry Pi and PC then rebooted.


\subsection{Hardware}
\label{\detokenize{productmaint:hardware}}
For physical maintenance it’s important to air dust the PC and RPi from time to time since they gather dust.
Instructions for opening up the PC and the RPi can be found on the Hardware Maintenance section of the hardware manufacturers websites.


\section{Troubleshooting}
\label{\detokenize{troubleshooting:troubleshooting}}\label{\detokenize{troubleshooting::doc}}

\subsection{Android-RPi Desktop Application Fails to Launch}
\label{\detokenize{troubleshooting:android-rpi-desktop-application-fails-to-launch}}\begin{itemize}
\item {} 
Ensure the Raspberry Pi is powered up

\item {} 
Ensure the Micro SD Card is seated correctly

\item {} 
Ensure the Ethernet Cable is corrected inserted (in both the Pi and the PC).

\item {} 
Reboot

\item {} 
Relaunch the Desktop Application

\end{itemize}


\subsection{Touchscreen Issues/ Interface is Frozen}
\label{\detokenize{troubleshooting:touchscreen-issues-interface-is-frozen}}\begin{itemize}
\item {} 
Reboot the Machine

\end{itemize}

After accepting an update or upgrade to the underlying operating systems on the devices may cause mild issues such as screen freezes.
Again these mild issues can be solved with a reboot. In most cases an upgrade or update won’t be performed if it is not compatible with the machines hardware.


\subsection{Unexpected Event}
\label{\detokenize{troubleshooting:unexpected-event}}
\sphinxstylestrong{Desktop Application is displaying the Android desktop on launch, not Loyverse application}

This may be a result of the startup manager application on Android being accidently removed. This creates a situation where Loyverse is not automatically opened during power up, which is why when you launch the desktop application it displays Android without the Loyverse application open. Kkeep notes of any changes you decide to make to the defaults, so that your custmisations and changes can be reversed later on if there’s a poor consequence.

\sphinxstylestrong{Window displaying Android/ Loyverse closes}

While the service is running you may see a command line window open in the background. If you close this window it closes the window displaying Android/Loyverse automatically because the fate of the two windows are interlinked. Alternatively the Raspberry Pi might have experienced a network failure (the ethernet cable has come loose) or a power loss or system failure has occured. The best resolution in all the instances is to check cables and connectors are properly seated and reboot the machine.


\section{Developers}
\label{\detokenize{developers:developers}}\label{\detokenize{developers::doc}}

\subsection{coming soon}
\label{\detokenize{developers:coming-soon}}
Hyperlink \sphinxhref{http://ArchLinuxarm.org/platforms/armv6/raspberry-pi}{here},


\subsection{command lines}
\label{\detokenize{developers:command-lines}}
like this: \sphinxcode{\sphinxupquote{192.168.0.x}}, \sphinxcode{\sphinxupquote{10.0.0.14x}} or

Enough of networking for now. We’ll set a proper network configuration later in this guide, but first some \sphinxstyleemphasis{musthaves}.


\subsubsection{text block}
\label{\detokenize{developers:text-block}}
\fvset{hllines={, ,}}%
\begin{sphinxVerbatim}[commandchars=\\\{\}]
\PYG{n}{passwd}  \PYG{c+c1}{\PYGZsh{} change root password to something important}
\PYG{n}{rm} \PYG{o}{\PYGZhy{}}\PYG{n}{rf} \PYG{o}{/}\PYG{n}{etc}\PYG{o}{/}\PYG{n}{localtime}  \PYG{c+c1}{\PYGZsh{} dont care about this}
\PYG{n}{ln} \PYG{o}{\PYGZhy{}}\PYG{n}{s} \PYG{o}{/}\PYG{n}{usr}\PYG{o}{/}\PYG{n}{share}\PYG{o}{/}\PYG{n}{zoneinfo}\PYG{o}{/}\PYG{n}{Europe}\PYG{o}{/}\PYG{n}{Prague} \PYG{o}{/}\PYG{n}{etc}\PYG{o}{/}\PYG{n}{localtime}  \PYG{c+c1}{\PYGZsh{} set appropriate timezone}
\PYG{n}{echo} \PYG{l+s+s2}{\PYGZdq{}}\PYG{l+s+s2}{my\PYGZus{}raspberry}\PYG{l+s+s2}{\PYGZdq{}} \PYG{o}{\PYGZgt{}}  \PYG{o}{/}\PYG{n}{etc}\PYG{o}{/}\PYG{n}{hostname}  \PYG{c+c1}{\PYGZsh{} set name of your RPi}

\PYG{n}{useradd} \PYG{o}{\PYGZhy{}}\PYG{n}{m} \PYG{o}{\PYGZhy{}}\PYG{n}{aG} \PYG{n}{wheel} \PYG{o}{\PYGZhy{}}\PYG{n}{s} \PYG{o}{/}\PYG{n}{usr}\PYG{o}{/}\PYG{n+nb}{bin}\PYG{o}{/}\PYG{n}{bash} \PYG{n}{common\PYGZus{}user} \PYG{c+c1}{\PYGZsh{}}
\PYG{n}{groupadd} \PYG{n}{webdata}  \PYG{c+c1}{\PYGZsh{} for sharing}
\PYG{n}{useradd} \PYG{o}{\PYGZhy{}}\PYG{n}{M} \PYG{o}{\PYGZhy{}}\PYG{n}{aG} \PYG{n}{webdata} \PYG{o}{\PYGZhy{}}\PYG{n}{s} \PYG{o}{/}\PYG{n}{usr}\PYG{o}{/}\PYG{n+nb}{bin}\PYG{o}{/}\PYG{n}{false} \PYG{n}{nginx}
\PYG{n}{usermod} \PYG{o}{\PYGZhy{}}\PYG{n}{aG} \PYG{n}{webdata} \PYG{n}{common\PYGZus{}user}

\PYG{n}{visudo}  \PYG{c+c1}{\PYGZsh{} uncomment this line:  \PYGZpc{}wheel ALL=(ALL) ALL}

\PYG{n}{pacman} \PYG{o}{\PYGZhy{}}\PYG{n}{Syu}
\end{sphinxVerbatim}

\sphinxstylestrong{bold text}
\begin{itemize}
\item {} 
bullet

\item {} 
point

\end{itemize}


\section{\sphinxstylestrong{Document Author(s):}}
\label{\detokenize{index:document-author-s}}

\subsection{Wave Telecom Limited}
\label{\detokenize{index:make-it-wave-ltd}}


\renewcommand{\indexname}{Index}
\printindex
\end{document}