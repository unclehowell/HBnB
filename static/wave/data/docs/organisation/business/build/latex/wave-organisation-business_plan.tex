%% Generated by Sphinx.
\def\sphinxdocclass{report}
\documentclass[letterpaper,10pt,openany,oneside,english]{sphinxmanual}
\ifdefined\pdfpxdimen
   \let\sphinxpxdimen\pdfpxdimen\else\newdimen\sphinxpxdimen
\fi \sphinxpxdimen=.75bp\relax

\PassOptionsToPackage{warn}{textcomp}
\usepackage[utf8]{inputenc}
\ifdefined\DeclareUnicodeCharacter
 \ifdefined\DeclareUnicodeCharacterAsOptional
  \DeclareUnicodeCharacter{"00A0}{\nobreakspace}
  \DeclareUnicodeCharacter{"2500}{\sphinxunichar{2500}}
  \DeclareUnicodeCharacter{"2502}{\sphinxunichar{2502}}
  \DeclareUnicodeCharacter{"2514}{\sphinxunichar{2514}}
  \DeclareUnicodeCharacter{"251C}{\sphinxunichar{251C}}
  \DeclareUnicodeCharacter{"2572}{\textbackslash}
 \else
  \DeclareUnicodeCharacter{00A0}{\nobreakspace}
  \DeclareUnicodeCharacter{2500}{\sphinxunichar{2500}}
  \DeclareUnicodeCharacter{2502}{\sphinxunichar{2502}}
  \DeclareUnicodeCharacter{2514}{\sphinxunichar{2514}}
  \DeclareUnicodeCharacter{251C}{\sphinxunichar{251C}}
  \DeclareUnicodeCharacter{2572}{\textbackslash}
 \fi
\fi
\usepackage{cmap}
\usepackage[T1]{fontenc}
\usepackage{amsmath,amssymb,amstext}
\usepackage[english]{babel}
\usepackage{times}
\usepackage[Bjarne]{fncychap}
\usepackage{sphinx}

\usepackage{geometry}

% Include hyperref last.
\usepackage{hyperref}
% Fix anchor placement for figures with captions.
\usepackage{hypcap}% it must be loaded after hyperref.
% Set up styles of URL: it should be placed after hyperref.
\urlstyle{same}

\addto\captionsenglish{\renewcommand{\figurename}{Fig.}}
\addto\captionsenglish{\renewcommand{\tablename}{Table}}
\addto\captionsenglish{\renewcommand{\literalblockname}{Listing}}

\addto\captionsenglish{\renewcommand{\literalblockcontinuedname}{continued from previous page}}
\addto\captionsenglish{\renewcommand{\literalblockcontinuesname}{continues on next page}}

\addto\extrasenglish{\def\pageautorefname{page}}

\setcounter{tocdepth}{1}



\title{Wave Business Plan}
\date{Feb 10, 2019}
\release{0.0.1}
\author{Author(s): Siôn H. Buckler, Wave}
\newcommand{\sphinxlogo}{\vbox{}}
\renewcommand{\releasename}{Release}
\makeindex

\begin{document}

\maketitle
\sphinxtableofcontents
\phantomsection\label{\detokenize{index::doc}}


\noindent\sphinxincludegraphics{{homehub}.png}


\chapter{Release Notes and Notices}
\label{\detokenize{releasenotes:release-notes-and-notices}}\label{\detokenize{releasenotes::doc}}
This section provides information about what is new or changed, including urgent issues, Software \& documentation updates, maintenance and new releases.
\begin{itemize}
\item {} 
‘Updates’ are the term used to describe significant changes to our public source code. These technical documents are now contained within our public source code.

\end{itemize}


\section{Version 0.2.0}
\label{\detokenize{releasenotes:version-0-2-0}}
The latest release/ draft of this business plan.


\subsection{Older Versions}
\label{\detokenize{releasenotes:older-versions}}
There are some older versions of this document. See the table below:


\begin{savenotes}\sphinxattablestart
\centering
\sphinxcapstartof{table}
\sphinxcaption{Table 1.0 - Older Versions of this Document}\label{\detokenize{releasenotes:id1}}
\sphinxaftercaption
\begin{tabular}[t]{|\X{25}{100}|\X{25}{100}|\X{25}{100}|\X{25}{100}|}
\hline
\sphinxstyletheadfamily 
archive date
&\sphinxstyletheadfamily 
version
&\sphinxstyletheadfamily 
description
&\sphinxstyletheadfamily 
download link
\\
\hline
2014-01-01
&
0.1.4
&
First Draft
&
\sphinxhref{https://wave.hotspotbnb.com/data/docs/organisation/business/build/html/\_static/archive/2014-01-01\_wave-business-plan\_version-0.1.4.pdf}{2014-01-01\_wave-business-plan\_version-0.1.4.pdf}
\\
\hline
2015-01-01
&
0.1.5
&
Updated Draft
&
\sphinxhref{https://wave.hotspotbnb.com/data/docs/organisation/business/build/html/\_static/archive/2015-01-01\_wave-business-plan\_version-0.1.5.pdf}{2015-01-01\_wave-business-plan\_version-0.1.5.pdf}
\\
\hline
2016-01-01
&
0.1.6
&
Updated Draft
&
\sphinxhref{https://wave.hotspotbnb.com/data/docs/organisation/business/build/html/\_static/archive/2016-01-01\_wave-business-plan\_version-0.1.6.pdf}{2016-01-01\_wave-business-plan\_version-0.1.6.pdf}
\\
\hline
\end{tabular}
\par
\sphinxattableend\end{savenotes}


\subsection{Version 0.1.6}
\label{\detokenize{releasenotes:version-0-1-6}}
2014 - 2017 Plan (Draft)


\subsection{Version 0.1.5}
\label{\detokenize{releasenotes:version-0-1-5}}
2016 - 2018 Plan (Draft)


\subsection{Version 0.1.4}
\label{\detokenize{releasenotes:version-0-1-4}}
2012 - 2014 Plan


\section{Known and Corrected Issues}
\label{\detokenize{releasenotes:known-and-corrected-issues}}\begin{description}
\item[{Below is a table of pending issues which have been reported to our team.}] \leavevmode
These issues will be cleared from this list as and when they are remedied.

\end{description}


\begin{savenotes}\sphinxattablestart
\centering
\sphinxcapstartof{table}
\sphinxcaption{Table 1.1 - Known Issues}\label{\detokenize{releasenotes:id2}}
\sphinxaftercaption
\begin{tabular}[t]{|\X{10}{100}|\X{10}{100}|\X{20}{100}|\X{60}{100}|}
\hline
\sphinxstyletheadfamily 
date
&\sphinxstyletheadfamily 
version
&\sphinxstyletheadfamily 
subject
&\sphinxstyletheadfamily 
description
\\
\hline
01-11-2018
&
0.0.1
&
N/A
&
no doubt many issues to report - first draft only
\\
\hline
\end{tabular}
\par
\sphinxattableend\end{savenotes}

\sphinxstylestrong{Comments} - none


\section{Recently Updated Topics}
\label{\detokenize{releasenotes:recently-updated-topics}}
Nothing significant to report


\chapter{Executive Statement}
\label{\detokenize{executive-statement:executive-statement}}\label{\detokenize{executive-statement::doc}}
Wave makes internet and energy a free birth-right for everyone.


\section{Business Overview}
\label{\detokenize{executive-statement:business-overview}}
Wave is a new kind of effort which makes internet and energy a free birth-right for everyone.
This charter presents real opportunity for global communities to fairer the distribution of their
resources; resulting in overall, improved living standards. Success in this field will mark the reach
of a new level in human development.

The previous level of human development was the industrial revolution. It allowed nations to
sustain. Prior to this we were in the agricultural age, where only cities and towns could be
sustained. Throughout all levels, a change in the economic system needed to occur in order it
evolve to be efficient enough for the new paradigm. The same economic shift is about to
happen as we near what is being deemed ‘the knowledge economy’. Although today’s advances
in technology warrant such a revolution, it hasn’t happened yet. Paradoxically, this is the 21st
Century’s greatest challenge, but also a substantial opportunity.

Wave was formed in 2012 by a platoon of communications and cyber security experts from the
Royal Corps of Signals. This community has financed the early phase research and development
of a pioneering new network telecommunications platform which aims to revolutionise the way
we trade internationally, maintain social order and even manage our earth’s resources.
The platform facilitates a new generation of applications which could serve as a primary
alternative to our archaic systems of law, currency and even government. The research findings
and early phase technology is already an innovative accomplishment, both technologically and
philosophically. But where the technology truly innovates is in its ability to access everybody and
not just a privileged few.

Not only is the technology de-centralised but it offers networks and advertising better than
anything in existence.

\sphinxstylestrong{Research}

Communities become as much as part of these systems as these systems become to
communities. Peoples trust and allegiance systematically shifts away from their instincts and
neighbours and into these archaic paper-based systems. The result is a very closed-off, selective
and controlling mindset. Agents of these systems regularly justify inhumane acts and work in
synchronicity to blockade, outlaw and enemize alternative social systems which threaten its
disempowerment.

Cyberspace is different. It brings us openness and transparency, making the entire environment
a power-leveller. Controversial virtual systems, such as BitCoin and WikiLeaks, have out-foxed
the fiat currency system and even our governmental system. Now Google’s Dr Eric Schmidt is
telling us to expect the emergence of a philosophical and technical alternative to our entire
system of law. The phrase he coined for this system is ‘Dual Crowdsourcing’. A virtual system
that allows anyone to report incidents up to the cloud and simultaneously allows jurors to pass
judgments down from the cloud. Collectively and anonymously in real-time.

Virtualised systems such as e-banking and government gateways seem to be methodical
enhancement of the old systems, but in actuality these cyber systems are just getting warmed
up. They are completely reinventing how we trade internationally, maintain social order and
manage resources. The new and old systems are already becoming notorious for locking horns
and contending at an ideological and philosophical level. As people move away from the old
systems and onto a new, we will witness one of the biggest global movement in over 5,000
years. A Cyber Civil War seems unfathomable but it will likely conclude in a matter of weeks or
even days. This is because of the immense traction that occurs in cyberspace. An example of this
is the game Angry Birds, which in 2012 gained 50 million subscribers in only 35 days. This is
more than the entire population of South Korea.

These examples show the very real possibility of a disruptive new system that could outperform
our system of money, law and government. And through the choice of the people, would also
outrank them. Since the emergence of social networks such as Facebook and Twitter, we’ve also
developed a completely new mindset to tradition of institutional thinking. Thomas Powers
coined the phrase ‘network thinking’, which is the trans-human behaviour of openness,
randomness and supportiveness. This supports a claim the above claim that a proposed digital
economical system such as WAVE, could find it fast and frictionless to find its way into our
everyday lives.

\sphinxstylestrong{Development}

The first order of business in to transform digital communications, from an affordable privilege
to an individual’s free and inherent right. The technology is now in development.

Digital, decentralised currency helps democratise the monetary system, but by its very nature it
will still cause inequality. An improvement in efficiency for our modern era would be a
moneyless system which can, in real time, measure and balance an individuals’ social
contribution with a calculation of their level of access to their desired resource.

Publicising legal cases and making the courts accessible via the web helps modernise the legal
system, but laws can still be unlawful, subject to interpretation and enforced unconstitutionally.
Our system of law could become more efficient with a computerised system which can
collectively and anonymously, collect information about an incident, then in real-time that same
system can facilitate the passing of a judgement from a collective and anonymous source
Wave is pioneering in the field of telecommunications and digital advertising. Our revolutionary
technology platform makes it possible for internet service providers and mobile operators to
offer its customers a free mobile communication device and unrestricted service plans.

We are developing the right technology for the right market at the right time. For an emerging
multi-£billion market opportunity. Our business team have planned and prepared an innovative
technology and business model that can meet this challenge head on and exploit the market
opportunity.

This concept will not only generate an extremely healthy ROI - it will also enable us to be at the
forefront of Corporate Social Responsibility at a global level, which will give our community the
opportunity to have an online voice and presence wherever and whoever they may be.

Making digital communications free is now technically possible and has now been proven with a
prototype technology developed by Wave Telecom. It eliminates the need to bill both home
and business users of any size for their internet usage as the high levels of targeted advertising
revenues mean we can not only cover these payments directly to the Internet Service Provider,
but also generate substantial profits after these costs are met.

This technology gained a firm position in several recent award ceremonies, including Cisco
British Innovation, Cyber Smart, Next Generation Digital Challenge and Digital Communications.
We are currently also entered into several awards for 2013 the outcomes of which shall be
announced later in the year.

During 2013 Wave Telecom Limited was incorporated and made principle trustee of the Extropy
Trust. The trust consists of a group of forward thinking telecommunications experts whom
research trends in future visions and finance the development of new era technologies.
This team has been developing, installing, maintaining and operating essential
telecommunications infrastructures and information systems for years, beginning their training
in The Royal Corps of Signals, which is now Britain’s most innovative force in cyber security and
defence in electronic warfare.

In accordance with the Trusts’ Charter, society’s biggest leap forward in the information age will
only occur once our ability to communicate digitally is transformed: “from an affordable
privilege to a free and inherent right”. It’s believed that this transformation alone will have
a more lasting positive impact than anything already experienced during the 21st
century.

Wave Telecom now manages the entire commercial operation and this incorporation marks the
beginning of the final stages of a concept that has been 2 years in development. We will not
only bring this concept to market but also develop the technologies that will facilitate it, both in
the short and long term. Once phone calls, internet access and the devices that enable these
benefits are made free to the individual, the entire world can interconnect at every level, all of
the time.

Development of these technologies in cyberspace and telecommunications can immediately
benefit everybody in the world without restriction or prejudice. The entire philosophy of law,
currency and government will be able to evolve, making these frameworks of our society more
lawful; promoting equality and self-governance.

So many of our day to day living expenses are on the increase to be able to not just reduce, but
fully eliminate, just one of these and enjoy the freedom to connect at zero cost would be
welcomed by both home and business users alike. But this is just the beginning.

Over the next 6 years we will be releasing some of the most innovative and state of the art
technology available in the world right now to the public in the UK and globally. Our aim is to
provide not only these technologies, but all other forms of communication such as mobile
phone use to our customers, cost free. We can also develop and provide our users with other
tools such as Wave social network, Wave search engine (with total cyber security), alternative
currency and social rating/scoring.

Our Wave technology development team is consistently 10 years ahead of their time with
theories, products, services and innovations. Instead of releasing these at a high cost to a select
few that can afford them - we will be turning this on its head and releasing to everyone for
zero cost

The products are available to anyone and cover the entire cross section of home and business
users. The list below shows just a few examples of our potential customers:


\section{Financial Overview}
\label{\detokenize{executive-statement:financial-overview}}

\section{Vision Statement}
\label{\detokenize{executive-statement:vision-statement}}

\section{Business Objectives}
\label{\detokenize{executive-statement:business-objectives}}
On release next year the process to sign up and use our product will be short and very simple.
- Go to the wave website www.wavetele.com
- Sign our terms and conditions
- Upload our software to any router
- Sign in through a Facebook account (until we offer our own “Wave” social network in
2015)
- Surf on the Wave platform

Wave Telecom will then meet the monthly bill costs for internet use by directly compensating
the ISP. The issues of wholesale cost and development of the payment procedure will be
agreed on with our Network Alliance ISP partners (see 3.2 “Types of Customer” for further
information on the Network Alliance).
On launch next year, all of these processes will have been rigorously tested by our
development team who will have created a seamless user experience.

The user will sign in initially through their Facebook account, but as we evolve our strategy is to
roll out our own search engine and social network. At the bottom of the screen the user will
see an unobtrusive banner on which targeted adverts will be shown and rotated every minute.

These adverts will be based on the user profile and therefore be relevant to the users’ product
or services preferences. An interactive online demo is available at www.wavetele.com/try.htm
These advertising campaigns, based on targeted relevant products and services, have been
trialled in case studies that form part of our previous research \& development campaigns over
the past two years. The details of which shall be provided within this document, but suffice to
say, the success rate is better than any other advertising campaign running on the internet.
Both Facebook and Google’s advertising campaigns give a maximum click through rate of 2\%.

The case studies of Wave Telecom produced a click through rate of up to 24\% - a figure not
reachable to any other campaign today.
Wave Telecom advertising is therefore extremely valuable and provides the business with a
highly lucrative revenue stream. The details of which are available in the 3 year projections
attached.

Our technology has already received approximately £300k in development funding, and we are
now ready to begin the next stage of progressing the company towards launching our products
and services to the public with a seed funding round of investment.

Wave Telecom: Business Objectives \& 3 Year UK Roll-out Strategy

Date Milestone
Aug-13 Seed Investment Raise: £20k for 1\% equity
Aug-13 Open Cardiff office x 4 person unit
Aug-13 Integration Alliance Launch: CENTIA porthole for collaboration on technical mandate \& Integration Alliance
Sales Executive to sign up relevant subscribers as per targets
Aug-13 Investment Manager to pitch to our network of Venture Capitalists
Aug-13 Marketing budget set @ 5\% of cash sale total: Year 1
Oct-13 Start-up Investment Round 1: £200k for 7\% equity stake
Jan-14 Start-up Investment Round 2: £200k for 6\% equity stake
Jan-14 Integration Alliance Project Manager
Feb-14 Patent submission
Feb-14 Wave Technical Mandate Tender Release \textendash{} Developer partners invited to tender
Feb-14 Open new Cardiff office x 9 person unit
Mar-14 Wave Platform Development: First Phase
Mar-14
Sales Manager \& Marketing Manager to prepare budgets, plans \& targets for year; technology output launch
programme; begin campaigns and raise awareness for pre-sales
Apr-14 Start-up Investment Round 3: £200k for 5\% equity stake
May-14 Wave Platform Development: Final Phase before release in June.
Jun-14 Open new Cardiff office x 32 person unit
Jun-14
Oct-14 Technology Output version 1.0 Wave Alpha: Wave Base, Wave Village, Wave District \& Wave Nation release to
CENTIA partners for testing phase and exclusive use pre and post launch
Jul-14 Marketing budget set @ 10\% of cash sale total: Year 2
Jul-14 Sales \& Marketing Plan \& Budget for Year 2
Jul-14 Increase sales \& marketing staff in preparation for version 2.0 release in October 2014
Aug-14 Start-up Investment Final Round 4: £250k for 4\% equity stake
Aug-14 Ad Inventory Account Manager close agreements with Ad partners for initial public launch
Technology Output version 2.0 Wave Alpha: Wave Base, Wave Village, Wave District \& Wave Nation release to
public. Following the testing phase the out-sourced suppliers will continue to work on any updates, changes
or on-going maintenance that the platform needs until they handover to the in-house development team in
April 2015. This also includes helpdesk and tech support for employees
Apr-15 Wave In-house Platform Development team take over development. Handover from out-source partner
Jul-15 Open new Cardiff office x 109 person unit
Aug-15 Technology Output version 2.1: Wave Search \& Wave Social Network
Jan-16 Technology Output version 2.2: App Store + App 1 (Dual crowdsourcing)
Jul-16 Technology Output version 2.3: Social Exchange + App 2 (Logarithmic Social Scoring)


\section{Opportunities \& Threats}
\label{\detokenize{executive-statement:opportunities-threats}}
Strengths
\begin{itemize}
\item {} 
Innovative business development team and mentors.

\end{itemize}

Established in Technology,
Investments, Scalable Online Services,
Marketing and NetworkTelecommunications.
- Strong community support \&
engagement
- Borderless, Free Internet Access and
Display Advertising Services
- 2 years R\&D, case studies \& traction
- Assumption proven correct
- £3.8million pre-trading valuation
- Young start-up - light and versatile

Weaknesses

Volatile young market
- High risk technology venture
- Future trend/assumption dependant
- Limited early resources;
\begin{quote}

o Pending start-up capital
o No dedicated developers
o Limited trading activity
\end{quote}
\begin{itemize}
\item {} \begin{description}
\item[{Limited Directors loans available}] \leavevmode
o Risk embracive history
o Limited capital

\end{description}

\item {} 
Heavy dependant on investment

\end{itemize}

Opportunities
\begin{itemize}
\item {} 
Build Alliances instead of competition

\item {} 
New standard in display advertising

\item {} 
Estimated £200Billion annual spend on display advertising by 2015

\item {} 
Meet government and industry

\end{itemize}

demand of interconnecting the worlds
remaining 4 billion population
- Free alternative for the connected 3
billion population
- Creates certainty amidst the uncertain
future of Cyberspace
- Platform for philanthropic and
experiment social applications
- Early sale of technology

Threats
\begin{itemize}
\item {} 
Competing tech giants

\item {} 
Privacy/ data security concerns

\item {} 
Bring unwelcomed social change

\item {} 
Political \& legislative ignorance

\item {} 
Irreversible social impact

\end{itemize}


\section{Exit Strategy}
\label{\detokenize{executive-statement:exit-strategy}}
Following the launch of all of our technology outputs in 2016, the exit strategy for Wave
Telecom Limited is obtainable through 3 different options. Throughout the trading period we
shall be advised by our company legal team and company accountant as to the best option to
take and the optimum period to do it within based on our ROI and profit margin for the year.

Option 1 \textendash{} Private Share Trading
Short term exit strategy Year 1-3
We have successfully offered our initial private investors the option to trade shares with new
investors as the business value increased through further investment into our technology and
R\&D. As promised, all our investors have traded whenever they have needed and are happy
with the ROI received, even at the early stages of the business. Our aim is to offer this service
throughout the first 3 years of trading.

Option 2 \textendash{} Merger/Acquisition/Friendly Takeover
Short term exit strategy Year 2-5
In an ideal world a perfect exit strategy for Wave would be to be part of a merger or acquisition
by another technology company or ISP. Our board of Directors would agree to this through a
public offer of stock or cash made by the acquiring firm, and the board of the target firm will
then publicly approve the buyout terms. Shareholder approval may be required dependent on
whether there any other shareholders other than directors at this point. The key determinant will
be whether the price per share offered is acceptable. Given the growth prospects of Wave, at
this point a premium figure would be expected.

Option 3 - Initial Public Offering (IPO)
Medium term exit strategy Year 3-5

“The first sale of stock by a private company to the general public on a securities exchange,
done by both smaller, younger companies seeking capital to expand or by largely privately
owned companies looking to become publicly traded”.
- Private Company -\textgreater{} Public Company
- Expansion capital raise
- Monetise investments of early private investors
- Become a publicly traded enterprise

Wave will employ the assistance of an underwriting firm (investment bank) to periodically advise
as to the best time to bring the company to market, the type of security to issue and the best
offering price. For those looking for a greater return from year 3 onwards, this may be a more
acceptable option.

Option 4 \textendash{} Alternative Investment Market (AIM)
Medium Term Exit Strategy Year 3-5
Another option would be to float our shares on the AIM once the business is generating a
healthy profit (from year 3 \textendash{} 5 onwards). This sub-market of the London Stock Exchange permits
smaller companies to participate with greater regulatory flexibility than applies to the main
market (no set requirements for capitalisation or number of shares issued). As it is also the
London Stock Exchange’s global market for smaller and growing companies it would suit Wave
as an exit option. Early stage businesses, venture capital-backed companies and more
established businesses may also join AIM, so we are not limited in stage or timescales within the
first 5 years.


\chapter{Business Summary}
\label{\detokenize{business-summary:business-summary}}\label{\detokenize{business-summary::doc}}
bla bla bla


\section{Business Details}
\label{\detokenize{business-summary:business-details}}
Wave Telecom Limited
Legal Status: Limited Company
Financial Year Start Month: April 2011


\section{Bank Details}
\label{\detokenize{business-summary:bank-details}}
Bank Account: Lloyds
Account Name: Wave Telecom Limited
Account Number: 23544668
Sort Code: 30-96-52


\section{Key People}
\label{\detokenize{business-summary:key-people}}
Sion Buckler


\section{Advisors}
\label{\detokenize{business-summary:advisors}}
OBE Penny \& Mr. Thomas Power
Winston Hamill
Sir Michael Moritz


\chapter{Market Analysis}
\label{\detokenize{market-analysis:market-analysis}}\label{\detokenize{market-analysis::doc}}
bla bla bla


\section{Market Research}
\label{\detokenize{market-analysis:market-research}}
Once upon a time, there were no computers, smart phones, or many of the other cool gadgets we now take for granted. Was there a need for computers a century ago? Did people feel they lacked something? Probably not. Newspapers brought most of the news. Writing was done by hand or on typewriters. When computers became commonplace, the technology was revolutionary, making the traditional instruments redundant. And now? Typewriters are a relic of the past. Most households consider having computers, smart phones, or other such devices as basic necessities.

The Internet has also become a necessity as a tool for reaching out on so many different levels; global communication, shopping, social networking, gaming, business, opinion, news, sport, music, video….the list is huge and so are the possibilities. It breaks down endless boundaries and gives us a freedom never before imagined - all from the comfort of your home, business or indeed, anywhere there is a decent Wi-Fi connection or hotspot.

According to Internet World Statistics the total online population in the UK was 52,731,209 users in June 2012 - this was equal to 84.1\% of the population. Furthermore, they cite the total Facebook user base in the UK as 32,950,400 people. This is our initial target market base as our users will log in using their social profile until Wave launches their own in Q3 2015*. This can also increase as customers may wish to create a Facebook account to gain access to the service we are offering i.e. free internet. These users we call our available online market.

There are 250,934,000 Facebook users in Europe and 725,009,960 users in the rest of the world. This gives an idea of the size of the available market that can be accessed once the service has been rolled out to Europe and the rest of the world. There is also room to grow our customer base as the internet penetration percentages in Europe and the Rest of the World show the available user base as more customers connect online. These users we call our emerging online market.

\sphinxstylestrong{UK Market}

The on-going financial crisis created by the banking system and poor government management has left most individuals and families paying the price in the UK. Our bills and cost of living are constantly on the increase, and taxes eat into every penny we earn and spend.

UK consumers spend approximately £1083 on line every year and the average yearly household broadband bill costs approximately £300 per year on data and line rental. That’s nearly a third of the total spend. In a UK population of over 63 million people, almost 53 million are online if, for example, a quarter of those people take up our free internet service we would have an available market of 13.25 million people with an extra £300 per year to spend where they choose.

We are not only enabling our customers to actually get on-line with no issue of cost, we are also enabling them to spend the additional disposable income available wherever needed. The likelihood is that some of those savings will be spent on-line on products and services. Regardless, it is certainly a substantial amount of money to spread throughout our economy and not just into the bulging pockets of the main ISP’s.

This equates to a win-win situation for everyone…our customers save money….they are free to spend these savings on both essentials and non-essentials on-line. They receive relevant advertising that meets their need - instead of, for example, generic; pushy campaigns run by no win no fee legal sharks. Advertisers receive valuable inventory which ultimately, sells products and services.

Not only that, but the economy receives a much needed boost as consumers have more spending power due to gaining an extra £20-£30 per month, for our business users, the monthly savings are even higher. Money movement such as this and the spreading of wealth more fairly is the ultimate business opportunity. Everyone gains and maybe even the economic outlook will look somewhat brighter too.

The business model is not entirely based on the free internet and advertising platform. This is just the start of a much bigger plan to make communications a safer and more socially beneficial experience to everyone. The internet has created a family who, if given the tools, can change things for the good and have a positive impact. Once the initial platform is tested and released we will add layers of innovative products and services that will be discussed in depth further into the business plan.

There are two themes which are central to the transformative effect of the internet on consumer behaviour and industry structures. The first is internet advertising revenue and the second is smartphone demand. As you can see from the key statistics the mobile advertising revenue is likely to grow quickly as the smartphone demand grows. Interestingly, users state they still use both mobile, tablet and laptop/PC devices equal to previous use depending on their location.

This is good news for advertising as not only is it growing on through other mediums - it is also holding its market share on older technologies. As the online audience has grown and more households have adopted faster internet connections \textendash{} allowing them to consume more content quicker- so businesses have increased spend on advertising to this audience.

\sphinxstylestrong{Internet Advertising}

“More revenue is generated by internet advertising that by any other sector”. In 2011, Internet advertising established itself as the largest ad category by spend and accounts for 30\% of advertising revenue in the UK - totalling approx. £4.8bn. Internet advertising is a key source of revenue for many of the online services consumers use.

Search advertising (£2.8bn) was the largest source of internet ad spend in 2011, followed by display (£1.1bn) and classified (0.8bn). Growth was fastest in search and display, up by 13\% and 18\% respectively on 2010 spend, while classified internet advertising has grown less quickly, up just 5\%.

Search advertising revenues are generated by adverts placed against specific keywords that internet users search for on search engines such as Google, Yahoo! And Bing. Search advertising is unique and allows advertisers to target specific users with specific interests. By contrast, internet display advertising is very similar to display advertising in press and elsewhere, except adverts are placed as banners on web pages rather than newspaper pages.

Wave targeted advertising is a hybrid of search advertising and display advertising. It uses the target advertising of search, through Facebook login customer/user profile and the banner advert style of display advertising, through the rotating banner at the bottom of the screen. This banner can be used to show both static and video imaging.

This widens our market reach, potential and available spend by meshing together the drivers of growth in internet advertising, which is in itself the largest area of advertising spend in the UK. The success of the click thru rate backs up the effectiveness of this type of advertising shown in case studies within this document (see page ?).

The strong growth of display and search can be accounted for in part by the emergence of 2 rapidly growing internet advertising markets: - Online video display advertising (revenue and time spend on film /video sited: 2008- 2011)

Video display advertising spend was £109m in 2011, approximately 10\% of all internet display advertising revenue. Video display ads have seen exceptional growth since 2008, increasing their revenue eight-fold.

Online video advertising can take one of two forms. The first is similar to display advertising on websites, but in the form of an audio-visual advert rather than a static image or a series of animated images. The second is similar to traditional spot television advertising, where adverts are shown either before, after or mid-way through an online video.

According to research published by comScore, just under 22 million internet users aged 6+ watched at least one online video advert in January 2012. Men were more likely than women to view online video adverts and the main age range for this was 15-24 year olds. Demographic profile of video advertisement viewers

In 2011, expenditure on mobile advertising rose to £203m, more than double that of 2010, and representing like-for-like growth of 157\%. Since 2008, mobile advertising revenues have grown 7-fold and in the same time the proportion of adults using their mobile phone to access the internet has doubled (20\% to 39\%).

The UK Communications Market Report 2011 highlighted the rise of the mobile internet and the likelihood that smartphone take-up was the primary driver of the increasing data services available on mobile phones.

It is also likely that smartphone take-up is the driver behind the growth of mobile advertising. Smartphones appear to facilitate the use of the mobile internet, increasing the mobile internet audience and making the platform a more attractive proposition for advertisers. Furthermore, the technological capabilities of today’s smartphones, such as touchscreens, large high definition displays, high-speed processors, and operating systems capable of installing apps, present a wider range of creative opportunities for advertisers.

Mobile advertising revenues and smartphone and mobile internet take-up.

In 2011 search is two-thirds (66\%) of mobile advertising revenue was generated by mobile search advertising, unchanged since 2010, and the remainder by mobile display (34\%). The advances in mobile handset technology represented by smartphones, and the move away from WAP, have increased the similarity of the mobile internet to the conventional PC internet. As such, mobile search adverts and mobile display adverts are very much the same as their counterparts described above

\sphinxstylestrong{Smartphone Demand}

“2 in every 5 UK adults now have a smartphone” Smartphones are a key enabler in the rise of the mobile internet and which has changed the way consumers live their everyday lives. Between 2011 and 2012 smartphone take up rose from 27\% to 39\% of UK adults, representing 43\% of mobile users. Smartphone take up is highest among younger age groups: 66\% of those aged 16 to 24 and 60\% of those aged 25 to 34 have a smartphone, as do 46\% of the ABC1

More than 4 in 10 smartphone users (42\%) agree with the statement: “my phone is more important to me for accessing the internet than any other device”. Agreement is highest in the 16-24 \& 25-44 age groups. Smartphone users are highly dependent on their phones. Just over 4 in 10 (41\%) smartphone users indicated high levels of addiction compared to 37\% in 2011.

Smartphones are used for traditional internet activities and new internet phenomena. The top 5 activities or functions used regularly on a smartphone by UK adults are e mail (51\%), internet surfing (44\%), social networking (42\%), taking photos/video (37\%), and listening to music (35\%). These are the same activities as recorded in the top five in the 2011 survey. Many of the activities have seen marginal increases since 2011, with e mail measuring the biggest increase, up 5\% from 46\% in 2011.

New activities or functions new to the 2012 survey are tweeting, checking into a place or location on social networks and using voice activated services. Smartphones are substituting for other devices and media formats. The activities that smartphone users claim their handset is substituting for most are: watching video clips on a PC or laptop, instant messaging from a PC or laptop and social networking from a PC or laptop. But significant numbers of people say that they are still doing the same amount of activities on other new devices since getting their smartphones.

“1 in 5 (20\%) of smartphone users also own a tablet PC” Tablet computers and smartphones have many features in common and despite the overlap in functionality, 1 in 5 smartphone users also own a PC. A significant proportion of consumers, for each activity, claimed to use their smartphone and tablet equally so it is likely that the device used is determined by the location of the consumer. Tablets are primarily used at home as a more portable; internet surfing \& user friendly version of bulkier and heavier laptop or PC. This is the evolution of our technology types based on needs of use. We are now seeing the introduction of the removable key pad (Asus Transformer Prime) that will thus give us 2 devices in 1. Tablet for play and keyboard option for work.

Smartphones are alternatively, used as a mobile device and as a private communication device; whereas two-thirds of consumers share their tablets in their household. Therefore, a tablet is less likely to be seen as a personal communications device. Principal device for selected activities among smartphone and tablet More than half (57\%) of smartphone users claim to have used their handset in some way when out shopping.

Key findings
- 8 in 10 households have access to the internet Household internet access rose to 80\% in Q1 2012, up 3\% points on the previous year.
- Over half of all UK households have 3 or more internet-enabled devices. 85\% of households own at least 1 internet enabled device and on average each household owns 3 different types of internet enabled device. Only 1 in a thousand owns all 10 devices surveyed.

Games consoles are more popular than laptops or PC’s among DE households. 46\% own a games console, compared to just 44\% who own a laptop and 30\% own a PC. In contrast, among AB households, 75\% own a laptop and 56\% own a PC, while 51\% own games consoles.
\begin{itemize}
\item {} 
Growth of accessing the internet on laptop and desktop computers is slowing.

\item {} 
Young adult men spend the most time online via a laptop or desktop computer (34.1 hours p/month) than any other age/gender group, and almost 10 hours more per month than the UK average of 24.2 hours for March 2012

\item {} 
Two thirds of 65-74 years olds now have home internet access, the largest rise among all age groups. This rose by 9\% to 64\% between 2012 to 2012. Internet access was highest among those aged 16 to 34 (90\%) and AB social groups (92\%).

\item {} 
1 in 7 UK adults do not intend to get the internet in the next year. Overall, 2 thirds cite lack of interest as the main reason for not getting the internet.

\end{itemize}

Wi-Fi networks are a key enabler for homes with multiple internet enabled devices !

In Q1 2012 the proportion of homes with broadband using a wireless router rose 10\% to 85\%. As highlighted in the graph above, a number of devices which were not widely available 5 years ago (e.g. netbooks, smartphones and tablets) can be connected to the internet over Wi-Fi

Internet-enabled devices

The networks over which the device connects and the internet experience that the device delivers both vary. The laptop is the most popular device that can connect to the internet among UK households (61\%). Games consoles are the second most popular type of internet enabled device, followed by desktop PC (44\%).

In regards to ownership of each type of internet-enabled device as a proportion of the AB, C1, C2 and DE socio-economic groups \textendash{} for almost all internet-enabled devices, ownership is highest among AB households and lowest among DE households, probably related to the greater disposable income to spend on such devices in AB households. The exception to this rule is ownership of TV-based and portable games console, where take up is higher among C1 and C2 households than in AB households.

Counter to the UK average, games consoles are more popular than laptops or PC’s among DE households, with PC’s and smartphones are more popular than games consoles among AB households.

The proportion of each social group owning internet-enabled devices

Three quarters of e-readers (74\%) and tablets (76\%) are owned by ABC1 households compared to just 6 in 10 internet-enabled set top boxes (Sky + etc.). Internet devices of all kinds are more likely to be owned by ABC1 households than C2DE households, but that balance of ownership varies according to device.

Recent market entrants such as e-readers, tablets and netbooks, the functionality of which is replicated among existing devices, are more likely to be owned by ABC1 households than C2DE households.

Those aged 16-24 were the most likely to have accessed the internet on a mobile phone, games console, or portable media player. However, those aged 25-34 and 35-44 were more likely than other age groups to have accessed the internet on a tablet computer. The proportion of households by social group in the 2001 UK census was AB 22\%, C1 29\%, C2 15\%, D 17\% and E16\%

Devices used to visit internet websites in 2011 by age. Each household in the UK has on average 3 different types of internet enabled device and 85\% of households have at least 1.

Number of different internet-enabled devices per household

\sphinxstylestrong{Internet take-up}

Home internet access in the UK continues to grow, increasing by 3\% to 80\% for Q1 2012. Home internet access is evenly spread across those aged 16 to 54 but decreases beyond this age range. However, the proportion of adults aged 16 to 54 but decreases beyond this age range. However, the proportion of adults aged 65 to 74 with home internet access grew strongly over the year, rising 9\% to 64\% by 2012. The \% of 16-24 year olds with access to the internet grew by 5\% between 2011-2012; to 90\%, and brings this age group level with those aged 25 to 34 (90\%) and 35 to 54 (88\%). Home internet access is highest for the AB socio-economic group (92\%) and lowest for DE’s (63\%), while access is 3\% higher among men (81\%) than women (78\%).

Home internet access by age, socio-economic group, and gender

Since January 2004 the size of the UK online audience rose on average 6.2\% each year; from 24.5 million to 39.7 million in January 2012 was 1.6\%, compared to a high of 10.3\% in January 2009. It is likely that the slowing growth of broadband take-up has contributed to the slowing growth of the online audience.

\sphinxstylestrong{Time Spent online}

In January 2012 the average amount of time internet users spent online through a laptop or desktop was 24.6 hours per month, more than double the amount of time users spent online in January 2004. However, the growth in time spent online appears to have plateaued. The average time online per month increased by only half an hour between 2010 and 2011; to 23.5 hours, and while increasing, this is still less than the peak average of 23.9 hours/moth in 2009. Two likely reasons for this slowed growth are the effect of late adopters and the growth of other internet-enabled devices. Late adopters of the internet characteristically spend less time online than average, so as more late adopters get connected the average time online may decrease. Furthermore, the data does not include time spent online on smartphones, tablets, or other internet-connected devices, which are likely to be substituting for time online on laptop or desktop computers.

33 million adults accessed the internet every day in 2012 in the UK, more than double the 2006 figure of 16 million. Approximately 87\% of adults aged between 16 and 24, used social networking sites in 2012, compared to 48\% of all adults. Telephone or video calls over the internet were made by 32\% of adults in 2012, double the 2009 estimate of 16\%, and four times higher than the 2007 estimate of 8\% (Office of National Statistics Feb 2013). The higher tier of earners in the socioeconomic status groups are at the front of the queue when it comes to technology. This “early adopter” behaviour tends to lead to earlier market penetration and advanced levels of usage, according to a recent eMarketer report. In 2012, data from Ofcom indicated that affluents spent an average 21 hours per week using the internet. This was well above the next most engaged socioeconomic demographic. But not only are UK affluents big users of digital media they are also wary of it, as their engagement with social media shows a low level of engagement. Those in the upper-middle and middle class use the internet for extremely varied purposes, with communication coming out on top: 92\% of affluents took part in this activity at least once a week, according to a survey by Ofcom. One activity that this group performed far more regularly than other groups was online transactions, with 61\% of affluents carrying out such transactions at least once per week. Smart mobile devices are another digital technology UK affluents have taken up quickly. According to Ofcom data, in terms of smartphones, the figure for affluents was 62\%, above all other segments. Tablets have also seen especially high uptake among affluents.

Summary:
- UK Consumers spend on average £1083 per year online.
- UK Population - 63,047,162
- UK Internet users June 30 2012 - 52,731,209
- Penetration - 83.6\%
- Tablet ownership in the UK has jumped from 2\% to 11\% in 12 months
- Facebook was the most searched for item on the web in all comparator countries (except Japan, Russia and China) to August 2012
- 16.4\% of internet traffic is generated by UK users accessing the internet through smartphones, tablets and other connected devices.
- Text based comms are surpassing traditional phone calls or meeting face to face as the most frequent way of keeping in touch
- In the UK 13 minutes out of every hour online is spent on social networking and forums, 9 minutes on entertainment and 6 minutes shopping. (Experian marketing services)
- UK emerged as having the most prolific online shoppers. They spent 10\% of all time online shopping in 2012.

GDK market research published results that in 2012 mobile device accounted for a total of 23\% of internet use. Previous year’s figures stood at 15\%. Convenience and accessibility of smartphones is clearly something which has won the confidence of the public and their use for online browsing will continue to grow.

Time spent online per day (approximate)
Mobile Browsing Desktop/Laptop Browsing
Browsing Internet 24 min Social Networks 15 min
Social Networks 16 min Search 4 min
Online Games 10 min
Classified/Auctions 7 min
Total 40 min 36 min

Total time spent online
Timescale Total Minutes Total Hours
Daily 76 minutes 1h 16min
Weekly 532 minutes 8h 52min
Monthly 2,128 minutes 35h 28min
Yearly 25,536 minutes 425h 36min


\section{Types of Customers}
\label{\detokenize{market-analysis:types-of-customers}}
The ideal customer profile is of someone who considers this saving relevant and meaningful, yet also has some spare income to make purchases on line and who also sees the advertising as useful and relevant. They will also be open to change and innovation and aware of social and environmental factors that are shaping the world at the moment.

If we break it down the ideal profile is as follows:
- Men and women
- Age 21-50
- Income: mainly lower-middle and higher-middle income individuals
- Singles, Married couples \& families
- Socially aware

Families, although may not have as much disposable income as singles \& married couples, are shorter on time and likely to use the internet to make purchases. The internet has a major advantage over the high street in that you can search and find the product for the lowest price available in minutes. Nobody these days wishes to spend hours walking or driving from shop to shop to find the best deal, when it can be accessed at home or at a hotspot almost instantly


\section{Types of Products and Services}
\label{\detokenize{market-analysis:types-of-products-and-services}}
Integration Alliance (CENTIA: Cyber Enhanced Networks and Telecommunications Integration Alliance) CENTIA is scheduled to launch in August 2013. This team of approximately 200 advertisers, ISP’s and developers will consist of some of the biggest movers \& shakers within the industry. The partners will all be charged a monthly subscription fee in return for input, exclusive use of the platform before public release, partnership deals and shares in the company. These partners will also have the opportunity to tender for any of the project mandates in the next few months and following release.

Pre sign up, each partner agree to be bound by our terms and conditions, which will include an NDA that will protect our business and any projects that are discussed and produced within CENTIA.

The alliance itself will stabilise and change con-currently with the business evolution. For example, as we move the focus from the launch of our free internet service to the development of the Wave App Store, the advertising agencies and ISP’s will unsubscribe and new, more relevant businesses and developers will take their place. These alternative partners will work with us to bring the next roll out to market and the WT customer base.

The number of members will stay constant throughout the lifespan of the business as the main objective is to keep bringing our customers the most innovative technology possible.


\section{Marketing and Promotion}
\label{\detokenize{market-analysis:marketing-and-promotion}}

\section{Main Competitors}
\label{\detokenize{market-analysis:main-competitors}}
This is currently an untapped niche market, which means competitors do not currently exist in the UK. Indeed, up until now, we have only found one other business that is operating within our sector in California but there are still major differences in our business models.


\section{Key Suppliers}
\label{\detokenize{market-analysis:key-suppliers}}

\chapter{Finanial Projections}
\label{\detokenize{financial-projections:finanial-projections}}\label{\detokenize{financial-projections::doc}}
bla bla bla


\section{Sales Forecast}
\label{\detokenize{financial-projections:sales-forecast}}

\section{Direct Costs}
\label{\detokenize{financial-projections:direct-costs}}

\section{Overheads}
\label{\detokenize{financial-projections:overheads}}

\section{Fixed Assets}
\label{\detokenize{financial-projections:fixed-assets}}

\section{Investments}
\label{\detokenize{financial-projections:investments}}

\section{Loans}
\label{\detokenize{financial-projections:loans}}

\section{Grants}
\label{\detokenize{financial-projections:grants}}

\chapter{Startup Costs and Funding}
\label{\detokenize{startup-funding:startup-costs-and-funding}}\label{\detokenize{startup-funding::doc}}
bla bla bla


\section{Startup}
\label{\detokenize{startup-funding:startup}}

\section{Series A}
\label{\detokenize{startup-funding:series-a}}

\chapter{Reports}
\label{\detokenize{reports:reports}}\label{\detokenize{reports::doc}}
bla bla bla


\section{Projected Cashflow}
\label{\detokenize{reports:projected-cashflow}}

\section{Projected Profit and Loss}
\label{\detokenize{reports:projected-profit-and-loss}}

\section{Projected Balance Sheet}
\label{\detokenize{reports:projected-balance-sheet}}

\chapter{\sphinxstylestrong{Document Author(s):}}
\label{\detokenize{index:document-author-s}}

\section{Siôn H. Buckler}
\label{\detokenize{index:sion-h-buckler}}

\begin{savenotes}\sphinxattablestart
\centering
\begin{tabulary}{\linewidth}[t]{|T|T|T|}
\hline
\sphinxstyletheadfamily 
Organisation
&\sphinxstyletheadfamily 
Role
&\sphinxstyletheadfamily 
Details
\\
\hline
\noindent\sphinxincludegraphics{{wave-logo}.png}
&
Founder \& CEO
&
Wave Telecom Limited, British Corporation (England \& Wales), Company Director ID 11363386
\\
\hline
\noindent\sphinxincludegraphics{{ccu}.png}
&
Head of Defence
&
Caribbean Communications Unit (CCU), Royal Corps of Signals, Life Member ID 55983
\\
\hline
\noindent\sphinxincludegraphics{{uarsociety1}.png}
&
Council President
&
Utilities as a Right (UaaR) Society, British Public Servant, Gov/Oath ID 25148537
\\
\hline
\noindent\sphinxincludegraphics{{scottishbay}.png}
&
Military Theorist
&
Scottish Bay, Dominican Republic (Green Line \& Treaty of Guarantee)
\\
\hline
\end{tabulary}
\par
\sphinxattableend\end{savenotes}

\sphinxstylestrong{About} Siôn Buckler - Science \& Computer Science (Bachelors), Electronic Engineering, Industrial Electronics and Electronics \& Computing (Advanced Diplomas), Cisco Certified Network Associate (CCNA), Microsoft Certified Solutions Expert (MCSE), Certified Project Management (Prince2 Practitioner), Institute of Electronic Engineering (IEEE), Siemens Certified Engineer, Certified Telecommunications Service Provider (NVQ3), Satellites \& Full Spectrum Radio, Fixed Telecommunications Systems with Enhanced Capabilities , SKP01 Electrical Safety, NVQ2 IT, Defence Specialist LAN, TCP/IP, Subnetting, DHCP, Addressing, Routing \& Browsing, Communications Equitment Room Design \& Maintenance, Military Command \& Leaderership,  Cyber Security, Electronic Warfare, SIP/ VOIP, SEO, PPC, HTML5, CSS3, Java, Perl, Ajax, JQuery, MySQL, Unix, Python, Linux.



\renewcommand{\indexname}{Index}
\printindex
\end{document}