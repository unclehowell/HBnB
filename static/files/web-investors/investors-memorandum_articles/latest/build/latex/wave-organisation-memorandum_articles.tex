%% Generated by Sphinx.
\def\sphinxdocclass{report}
\documentclass[letterpaper,10pt,openany,oneside,english]{sphinxmanual}
\ifdefined\pdfpxdimen
   \let\sphinxpxdimen\pdfpxdimen\else\newdimen\sphinxpxdimen
\fi \sphinxpxdimen=.75bp\relax

\PassOptionsToPackage{warn}{textcomp}
\usepackage[utf8]{inputenc}
\ifdefined\DeclareUnicodeCharacter
 \ifdefined\DeclareUnicodeCharacterAsOptional
  \DeclareUnicodeCharacter{"00A0}{\nobreakspace}
  \DeclareUnicodeCharacter{"2500}{\sphinxunichar{2500}}
  \DeclareUnicodeCharacter{"2502}{\sphinxunichar{2502}}
  \DeclareUnicodeCharacter{"2514}{\sphinxunichar{2514}}
  \DeclareUnicodeCharacter{"251C}{\sphinxunichar{251C}}
  \DeclareUnicodeCharacter{"2572}{\textbackslash}
 \else
  \DeclareUnicodeCharacter{00A0}{\nobreakspace}
  \DeclareUnicodeCharacter{2500}{\sphinxunichar{2500}}
  \DeclareUnicodeCharacter{2502}{\sphinxunichar{2502}}
  \DeclareUnicodeCharacter{2514}{\sphinxunichar{2514}}
  \DeclareUnicodeCharacter{251C}{\sphinxunichar{251C}}
  \DeclareUnicodeCharacter{2572}{\textbackslash}
 \fi
\fi
\usepackage{cmap}
\usepackage[T1]{fontenc}
\usepackage{amsmath,amssymb,amstext}
\usepackage[english]{babel}
\usepackage{times}
\usepackage[Bjarne]{fncychap}
\usepackage{sphinx}

\usepackage{geometry}

% Include hyperref last.
\usepackage{hyperref}
% Fix anchor placement for figures with captions.
\usepackage{hypcap}% it must be loaded after hyperref.
% Set up styles of URL: it should be placed after hyperref.
\urlstyle{same}

\addto\captionsenglish{\renewcommand{\figurename}{Fig.}}
\addto\captionsenglish{\renewcommand{\tablename}{Table}}
\addto\captionsenglish{\renewcommand{\literalblockname}{Listing}}

\addto\captionsenglish{\renewcommand{\literalblockcontinuedname}{continued from previous page}}
\addto\captionsenglish{\renewcommand{\literalblockcontinuesname}{continues on next page}}

\addto\extrasenglish{\def\pageautorefname{page}}

\setcounter{tocdepth}{1}



\title{Wave Memorandum and Articles}
\date{Dec 09, 2019}
\release{0.1.1}
\author{Author(s): Wave Telecom Limited}
\newcommand{\sphinxlogo}{\vbox{}}
\renewcommand{\releasename}{Release}
\makeindex

\begin{document}

\maketitle
\sphinxtableofcontents
\phantomsection\label{\detokenize{index::doc}}


The Companies Act 2006. Private Company Limited by shares.

Memorandum of Association of; \sphinxstylestrong{Wave Telecom Limited}

Company Number; Pending

Each subscriber to this memorandum of association wishes to form a company under the Companies Act 2006 and agrees to become a member of the company and to take at least one share

Names of subscriber(s); \sphinxstylestrong{SION HYWEL BUCKLER}

Dated: 22/10/2019


\chapter{Release Notes and Notices}
\label{\detokenize{releasenotes:release-notes-and-notices}}\label{\detokenize{releasenotes::doc}}
This section provides information about what is new or changed, including urgent issues, Software \& documentation updates, maintenance and new releases.
\begin{itemize}
\item {} 
‘Updates’ are the term used to describe significant changes to our public source code. These technical documents are now contained within our public source code.

\end{itemize}


\section{Version 0.1.1}
\label{\detokenize{releasenotes:version-0-1-1}}\begin{itemize}
\item {} 
Changed company from \sphinxstylestrong{Wave Telecom Limited} to \sphinxstylestrong{Wave Telecom Limited}

\item {} 
This document isn’t finished, will need to return and complete - In the interim there’s a downloadable copy of the document from the formation agent in the older versions table below (dated 2019-10-21).

\item {} 
The version control just leaped from 0.0.1 to 0.1.1. So this version is 0.1.1 and the previous version is 0.0.1, but moving forward the previous version will be refered to as 0.1.0. The archieved version (0.1.0) still refers to itself as version 0.0.1, since archieved files aren’t altered after being archieved.

\end{itemize}


\subsection{Older Versions}
\label{\detokenize{releasenotes:older-versions}}
See the table below for downloadable links to older versions of this document:


\begin{savenotes}\sphinxattablestart
\centering
\sphinxcapstartof{table}
\sphinxcaption{Table 1.0 - Older Versions of this Document}\label{\detokenize{releasenotes:id1}}
\sphinxaftercaption
\begin{tabular}[t]{|\X{25}{100}|\X{25}{100}|\X{25}{100}|\X{25}{100}|}
\hline
\sphinxstyletheadfamily 
archive date
&\sphinxstyletheadfamily 
version
&\sphinxstyletheadfamily 
description
&\sphinxstyletheadfamily 
download link
\\
\hline
2019-10-21
&
N/A
&
Wave Telecom Limited
&
\sphinxurl{https://wave.hotspotbnb.com/data/organisation/memorandum-articles/build/html/\_static/archive/2019-10-21\_wave-memorandum-articles.pdf}
\\
\hline
2018-05-14
&
0.1.0
&
Wave Telecom Limited
&
\sphinxurl{https://wave.hotspotbnb.com/data/organisation/memorandum-articles/build/html/\_static/archive/2019-05-14\_wave-memorandum-articles\_v0.1.0.pdf}
\\
\hline
\end{tabular}
\par
\sphinxattableend\end{savenotes}


\subsection{Version 0.1.0}
\label{\detokenize{releasenotes:version-0-1-0}}
This is the first release/ draft of this technical document.


\section{Known and Corrected Issues}
\label{\detokenize{releasenotes:known-and-corrected-issues}}\begin{description}
\item[{Below is a table of pending issues which have been reported to our team.}] \leavevmode
These issues will be cleared from this list as and when they are remedied.

\end{description}


\begin{savenotes}\sphinxattablestart
\centering
\sphinxcapstartof{table}
\sphinxcaption{Table 1.1 - Known Issues}\label{\detokenize{releasenotes:id2}}
\sphinxaftercaption
\begin{tabular}[t]{|\X{10}{100}|\X{10}{100}|\X{20}{100}|\X{60}{100}|}
\hline
\sphinxstyletheadfamily 
date
&\sphinxstyletheadfamily 
version
&\sphinxstyletheadfamily 
subject
&\sphinxstyletheadfamily 
description
\\
\hline
2018-05-14
&
0.1.0
&
N/A
&
no doubt many issues to report - first draft only
\\
\hline
\end{tabular}
\par
\sphinxattableend\end{savenotes}

\sphinxstylestrong{Comments} - none


\section{Recently Updated Topics}
\label{\detokenize{releasenotes:recently-updated-topics}}
Nothing significant to report


\chapter{Introduction}
\label{\detokenize{introduction:introduction}}\label{\detokenize{introduction::doc}}

\section{Interpretation and Limitation of Liability}
\label{\detokenize{introduction:interpretation-and-limitation-of-liability}}

\subsection{Article 01: Defined Terms}
\label{\detokenize{introduction:article-01-defined-terms}}
In the articles, unless the context requires otherwise:

\sphinxstylestrong{“articles”} - means the company’s articles of association;

\sphinxstylestrong{“bankruptcy”} - includes individual insolvency proceedings in a jurisdiction other than England and Wales or Northern Ireland which have an effect similar to that of bankruptcy;

\sphinxstylestrong{“Secretary General”} - has the meaning given in article 12;

\sphinxstylestrong{“Secretary General of the meeting”} - has the meaning given in article 39;

\sphinxstylestrong{“Companies Acts”} - means the Companies Acts (as defined in section 2 of the Companies Act 2006), in so far as they apply to the company;

\sphinxstylestrong{“director”} - means a director of the company, and includes any person occupying the position of director, by whatever name called;

\sphinxstylestrong{“distribution recipient”} - has the meaning given in article 31;

\sphinxstylestrong{“document”} - includes, unless otherwise specified, any document sent or supplied in electronic form;

\sphinxstylestrong{“electronic form”} - has the meaning given in section 1168 of the Companies Act 2006;

\sphinxstylestrong{“fully paid”} - in relation to a share, means that the nominal value and any premium to be paid to the company in respect of that share have been paid to the company;

\sphinxstylestrong{“hard copy form”} - has the meaning given in section 1168 of the Companies Act 2006;

\sphinxstylestrong{“holder”} - in relation to shares means the person whose name is entered in the register of members as the holder of the shares;

\sphinxstylestrong{“instrument”} - means a document in hard copy form;

\sphinxstylestrong{“ordinary resolution”} - has the meaning given in section 282 of the Companies Act 2006;

\sphinxstylestrong{“paid”} - means paid or credited as paid;

\sphinxstylestrong{“participate”} - in relation to a directors’ meeting, has the meaning given in {\hyperref[\detokenize{directors:article-10}]{\sphinxcrossref{\DUrole{std,std-ref}{Article 10: Participation in Directors’ Meetings}}}};

\sphinxstylestrong{“proxy notice”} has the meaning given in article 45;

\sphinxstylestrong{“shareholder”} means a person who is the holder of a share;

\sphinxstylestrong{“shares”} means shares in the company;

\sphinxstylestrong{“special resolution”} has the meaning given in section 283 of the Companies Act 2006;

\sphinxstylestrong{“subsidiary”} has the meaning given in section 1159 of the Companies Act 2006;

\sphinxstylestrong{“transmittee”} means a person entitled to a share by reason of the death or bankruptcy of a shareholder or otherwise by operation of law;

\sphinxstylestrong{“writing”} means the representation or reproduction of words, symbols or other information in a visible form by any method or combination of methods, whether sent or supplied in electronic form or otherwise.

Unless the context otherwise requires, other words or expressions contained in these articles bear the same meaning as in the Companies Act 2006 as in force on the date when these articles become binding on the company.


\subsection{Article 02: Liability of Members}
\label{\detokenize{introduction:article-02-liability-of-members}}
The liability of the members is limited to the amount, if any, unpaid on the shares held by them.


\chapter{Directors}
\label{\detokenize{directors:directors}}\label{\detokenize{directors::doc}}

\section{Directors’ Powers and Responsibilities}
\label{\detokenize{directors:directors-powers-and-responsibilities}}

\subsection{Article 3: Directors’ General Authority}
\label{\detokenize{directors:article-3-directors-general-authority}}\label{\detokenize{directors:article-3}}
Subject to the articles, the directors are responsible for the management of the company’s business, for which purpose they may exercise all the powers of the company


\subsection{Article 4: Shareholders’ Reserve Power}
\label{\detokenize{directors:article-4-shareholders-reserve-power}}\label{\detokenize{directors:article-4}}
\sphinxtitleref{1.} The shareholders may, by special resolution, direct the directors to take, or refrain from taking, specified action

\sphinxtitleref{2.} No such special resolution invalidates anything which the directors have done before the passing of the resolution


\subsection{Article 5: Directors May Delegate}
\label{\detokenize{directors:article-5-directors-may-delegate}}\label{\detokenize{directors:article-5}}
\sphinxtitleref{1.} Subject to the articles, the directors, may,as they think fit, delegate any of the powers which are conferred on them under the articles
\begin{quote}

\sphinxtitleref{a.} to such person or committee;

\sphinxtitleref{b.} by such means (including by power of attorney);

\sphinxtitleref{c.} to such an extent;

\sphinxtitleref{d.} in relation to such matters or territories; and

\sphinxtitleref{e.} on such terms and conditions;
\end{quote}

\sphinxtitleref{2.} If the directors so specify, any such delegation may authorise further delegation of the directors’ powers by any person to whom they are delegated.

\sphinxtitleref{3.} The directors may revoke any delegation in whole or part, or alter its terms and conditions.


\subsection{Article 6: Committees}
\label{\detokenize{directors:article-6-committees}}\label{\detokenize{directors:article-6}}
\sphinxtitleref{1.} Committees to which the directors delegate any of their powers must follow procedures which are based as far as they are applicable on those provisions of the articles which govern the taking of decisions by directors

\sphinxtitleref{2.} The directors may make rules of procedure for all or any committees, which prevail over rules derived from the articles if they are not consistent with them


\section{Decisionmaking by Directors}
\label{\detokenize{directors:decisionmaking-by-directors}}

\subsection{Article 7: Directors To Take Decisions Collectively}
\label{\detokenize{directors:article-7-directors-to-take-decisions-collectively}}\label{\detokenize{directors:article-7}}
\sphinxtitleref{1.} The general rule about decisionmaking by directors is that any decision of the directors must be either a majority decision at a meeting or a decision taken in accordance with {\hyperref[\detokenize{directors:article-8}]{\sphinxcrossref{\DUrole{std,std-ref}{article 8}}}}

\sphinxtitleref{2.} If
\begin{quote}

\sphinxtitleref{a.} the company only has one director, and

\sphinxtitleref{b.} no provision of the articles requires it to have more than one director,
\end{quote}

the general rule does not apply, and the director may take decisions without regard to any of the provisions of the articles relating to directors’ decisionmaking.


\subsection{Article 8: Unanimous Decisions}
\label{\detokenize{directors:article-8-unanimous-decisions}}\label{\detokenize{directors:article-8}}
\sphinxtitleref{1.} A decision of the directors is taken in accordance with this article when all eligible directors indicate to each other by any means that they share a common view on a matter.

\sphinxtitleref{2.} Such a decision may take the form of a resolution in writing, copies of which have been signed by each eligible director or to which each eligible director has otherwise indicated agreement in writing

\sphinxtitleref{3.} References in this article to eligible directors are to directors who would have been entitled to vote on the matter had it been proposed as a resolution at a directors’ meeting.

\sphinxtitleref{4.} A decision may not be taken in accordance with this article if the eligible directors would not have formed a quorum at such a meeting.


\subsection{Article 9: Calling a Directors Meeting}
\label{\detokenize{directors:article-9-calling-a-directors-meeting}}\label{\detokenize{directors:article-9}}
\sphinxtitleref{1.} Any director may call a directors’ meeting by giving notice of the meeting to the directors or by authorising the company secretary (if any) to give such notice.

\sphinxtitleref{2.} Notice of any directors’ meeting must indicate
\begin{quote}

\sphinxtitleref{a.} its proposed date and time;

\sphinxtitleref{b.} where it is to take place; and

\sphinxtitleref{c.} if it is anticipated that directors participating in the meeting will not be in the same place, how it is proposed that they should communicate with each other during the meeting.
\end{quote}

\sphinxtitleref{3.} Notice of a directors’ meeting must be given to each director , but need not be in writing

\sphinxtitleref{4.} Notice of a directors’ meeting need not be given to directors who waive their entitlement to notice of that meeting, by giving notice to that effect to the company not more than 7 days after the date on which the meeting is held. Where such notice is given after the meeting has been held, that does not affect the validity of the meeting, or of any business conducted at it.


\subsection{Article 10: Participation in Directors’ Meetings}
\label{\detokenize{directors:article-10-participation-in-directors-meetings}}\label{\detokenize{directors:article-10}}
\sphinxtitleref{1.} Subject to the articles, directors participate in a directors’ meeting, or part of a directors’ meeting, when
\begin{quote}

\sphinxtitleref{a.} the meeting has been called and takes place in accordance with the articles, and

\sphinxtitleref{b.} they can each communicate to the others any information or opinions they have on any particular item of the business of the meeting.
\end{quote}

\sphinxtitleref{2.} In determining whether directors are participating in a directors’ meeting, it is irrelevant where any director is or how they communicate with each other.

\sphinxtitleref{3.} If all the directors participating in a meeting are not in the same place, they may decide that the meeting is to be treated as taking place wherever any of them is.


\subsection{Article 11: Quorum for Directors’ Meetings}
\label{\detokenize{directors:article-11-quorum-for-directors-meetings}}\label{\detokenize{directors:article-11}}
\sphinxtitleref{1.} At a directors’ meeting, unless a quorum is participating, no proposal is to be voted on, except a proposal to call another meeting.

\sphinxtitleref{2.} The quorum for directors’ meetings may be fixed from time to time by a decision of the directors, but it must never be less than two, and unless otherwise fixed it is two.

\sphinxtitleref{3.} If the total number of directors for the time being is less than the quorum required, the directors must not take any decision other than a decision
\begin{quote}

\sphinxtitleref{a.} to appoint further directors, or

\sphinxtitleref{b.} to call a general meeting so as to enable the shareholders to appoint further directors.
\end{quote}


\subsection{Article 12: Chairing of Directors’ Meetings}
\label{\detokenize{directors:article-12-chairing-of-directors-meetings}}\label{\detokenize{directors:article-12}}
\sphinxtitleref{1.} The directors may appoint a director to chair their meetings.

\sphinxtitleref{2.} The person so appointed for the time being is known as the Secretary General.

\sphinxtitleref{3.} The directors may terminate the Secretary General’ s appointment at any time.

\sphinxtitleref{4.} If the Secretary General is not participating in a directors’ meeting within ten minutes of the time at which it was to start, the participating directors must appoint one of themselves to chair it.


\subsection{Article 13: Casting Vote}
\label{\detokenize{directors:article-13-casting-vote}}\label{\detokenize{directors:article-13}}
\sphinxtitleref{1.} If the numbers of votes for and against a proposal are equal, the Secretary General or other director chairing the meeting has a casting vote.

\sphinxtitleref{2.} But this does not apply if, in accordance with the articles, the Secretary General or other director is not to be counted as participating in the decision-making process for quorum or voting purposes.


\subsection{Article 14: Conflict of Interest}
\label{\detokenize{directors:article-14-conflict-of-interest}}\label{\detokenize{directors:article-14}}
\sphinxtitleref{1.} If a proposed decision of the directors is concerned with an actual or proposed transaction or arrangement with the company in which a director is interested, that director is not to be counted as participating in the decision-making process for quorum or voting purposes.

\sphinxtitleref{2.} But if paragraph (3) applies, a director who is interested in an actual or proposed transaction or arrangement with the company is to be counted as participating in the decision-making process for quorum and voting purposes.

\sphinxtitleref{3.} This paragraph applies when
\begin{quote}

\sphinxtitleref{a.} the company by ordinary resolution disapplies the provision of the articles which would otherwise prevent a director from being counted as participating in the decision-making process;

\sphinxtitleref{b.} the director’s interest cannot reasonably be regarded as likely to give rise to a conflict of interest; or

\sphinxtitleref{c.} the director’s conflict of interest arises from a permitted cause.
\end{quote}

\sphinxtitleref{4.} For the purposes of this article, the following are permitted causes
\begin{quote}

\sphinxtitleref{a.} a guarantee given, or to be given, by or to a director in respect of an obligation incurred by or on behalf of the company or any of its subsidiaries;

\sphinxtitleref{b.} subscription, or an agreement to subscribe, for shares or other securities of the company or any of its subsidiaries, or to underwrite, sub-underwrite, or guarantee subscription for any such shares or securities; and

\sphinxtitleref{c.} arrangements pursuant to which benefits are made available to employees and directors or former employees and directors of the company or any of its subsidiaries which do not provide special benefits for directors or former directors.
\end{quote}

\sphinxtitleref{5.} For the purposes of this article, references to proposed decisions and decision-making processes include any directors’ meeting or part of a directors’ meeting.

\sphinxtitleref{6.} Subject to paragraph (7), if a question arises at a meeting of directors or of a committee of directors as to the right of a director to participate in the meeting (or part of the meeting) for voting or quorum purposes, the question may, before the conclusion of the meeting, be referred to the Secretary General whose ruling in relation to any director other than the Secretary General is to be final and conclusive.

\sphinxtitleref{7.} If any question as to the right to participate in the meeting (or part of the meeting) should arise in respect of the Secretary General, the question is to be decided by a decision of the5directors at that meeting, for which purpose the Secretary General is not to be counted as participating in the meeting (or that part of the meeting) for voting or quorum purposes.


\subsection{Article 15: Records of Decisions to be Kept}
\label{\detokenize{directors:article-15-records-of-decisions-to-be-kept}}\label{\detokenize{directors:article-15}}
The directors must ensure that the company keeps a record, in writing, for at least 10 years from the date of the decision recorded, of every unanimous or majority decision taken by the directors.


\subsection{Article 16: Directors’ Discretion to make Further Rules}
\label{\detokenize{directors:article-16-directors-discretion-to-make-further-rules}}\label{\detokenize{directors:article-16}}
Subject to the articles, the directors may make any rule which they think fit about how they take decisions, and about how such rules are to be recorded or communicated to directors.


\section{Appointment of Directors}
\label{\detokenize{directors:appointment-of-directors}}

\subsection{Article 17: Methods of Appointing Directors}
\label{\detokenize{directors:article-17-methods-of-appointing-directors}}\label{\detokenize{directors:article-17}}

\chapter{Shares and Distribution}
\label{\detokenize{shares:shares-and-distribution}}\label{\detokenize{shares::doc}}

\section{Shares}
\label{\detokenize{shares:shares}}

\subsection{Article 21: All Shares to be Fully Paid Up}
\label{\detokenize{shares:article-21-all-shares-to-be-fully-paid-up}}

\chapter{Decision-Making by Shareholders}
\label{\detokenize{decisionmaking:decision-making-by-shareholders}}\label{\detokenize{decisionmaking::doc}}

\section{Organisation of General Meetings}
\label{\detokenize{decisionmaking:organisation-of-general-meetings}}

\subsection{Article 37: Attendance and Speaking at General Meetings}
\label{\detokenize{decisionmaking:article-37-attendance-and-speaking-at-general-meetings}}

\chapter{Administative Arrangements}
\label{\detokenize{administrativearrangements:administative-arrangements}}\label{\detokenize{administrativearrangements::doc}}

\section{Final Articles}
\label{\detokenize{administrativearrangements:final-articles}}

\subsection{Article 48: Means of Communications to be Used}
\label{\detokenize{administrativearrangements:article-48-means-of-communications-to-be-used}}

\chapter{\sphinxstylestrong{Document Author(s):}}
\label{\detokenize{index:document-author-s}}
\sphinxstylestrong{Wave Telecom Limited}, Hampshire, GB

and

\sphinxstylestrong{Wave Telecom Limited} …



\renewcommand{\indexname}{Index}
\printindex
\end{document}